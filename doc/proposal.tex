\documentclass{acm_proc_article-sp}

\begin{document}

\title{Filesystem V2}
\subtitle{[A proposal for a C++ standard library]}

\numberofauthors{1}
\author{
\alignauthor
Eric Fiselier
    \affaddr{University of Calgary}
    \email{eric@efcs.ca}
}
\maketitle

\begin{abstract}
    The standardization of C++ libraries in a vital part of language evolution.
    One of the mostly widely used and requested is the filesystem library. The
    current filesystem library proposal is based of the boost filesystem library
    and suffers from a range of security issue. The goal of this research project
    is to produce a second revision of the filesystem proposal and submit it for
    standardization.

    The current filesystem proposal is N3505
    \footnote{http://www.open-std.org/jtc1/sc22/wg21/ docs/papers/2013/n3505.html}.

\end{abstract}

\section{The Problem}
    "Time of check to time of use" is the main problem this proposal will try
    and address. In filesystem v1 strings are used to represent paths. Since
    these strings have very little relation to actual filesystem entities this
    makes the entire design susceptible to TOCTOU race conditions. Similar
    problems exist when checking filesystem permissions and other information.
    \footnote{See the status function}
    This renders the entire design insecure and prone to race conditions.

    The second problem with the current proposal is its testability. The current
    design ties the library API with the underlying OS API directly. For
    example, the \verb&resize_file& function is specified in terms of the POSIX
    \verb&truncate& function. This makes it impossible to test how \verb&resize_file&
    handles improbable error conditions like "disk full".

    The API linkage described above also leads to a third problem. Since the API
    is specified in terms of POSIX function calls, it makes it hard to implement
    on non-POSIX systems. It also makes the current design unusable on systems
    with multiple filesystems.

\section{Motivation}
    This work is motivated by discussions with parties representing Google and
    IBM. While the current filesystem proposal covers a wide swath of use cases
    it omits consideration of uses in more "industrial" settings. In these settings
    the aforementioned concerns regarding security and irregular filesystems
    make the current proposal unusable.

    Furthermore, standardizing a library that promotes TOCTOU race conditions will
    promote end users to write insecure code. Previously standardized
    functions like \verb&rand&\footnote{http://channel9.msdn.com/Events/GoingNative/2013/rand-Considered-Harmful}
    and \verb&gets&\footnote{http://en.cppreference.com/w/c/io/gets}
    have caused non-trivial real world security vulnerabilities. Obviously it 
    would be greatly beneficial to standardize a design that is not prone to
    these problems.

    Also increasing the testability of the design will greatly the quality of
    implementation. The current design is impossible to test without actually
    manipulating the underlying systems filesystem. This limits the abilities of
    implementations to write and ship safe and portable test suites. For example
    nobody wants the test case for recursively removing a directory to wipe away
    an end users entire filesystem. The ability to test against faux filesystems
    will greatly help implementers.

\section{A Solution}
    With some difficulty I will try and describe the direction for the potential
    solution. It's important to remember that the solution must be all
    encompassing. ie. it must not only solve the problem but also provide API
    design that is both useful and implementable. For this reason describing the solution
    reduces to actually solving the problem.

    \subsection {Solving TOCTOU}
    The main idea behind solving the TOCTOU problem is designing (and limiting)
    the API to provide only "atomic" function. The idea behind atomic functions is
    simple. Looking up a file or asking for information about it should be
    equivalent to opening it. By doing so you lock the information returned to
    a concrete file handle. While this limits the functionality the API can
    provide it makes the API as a whole more secure.

    \subsection {Solving Testability and Multiple Filesystem Support}
    The solutions to these problems is to switch the design to using a service
    provider model. There will be a filesystem object that will accept a service
    provider upon construction. This allows for users to provide mock filesystems
    as well as to provide custom service providers for irregular filesystems.

    The service provider will be modeled in term of a set of concepts. The 
    implementation will depend on the provider to implement these concepts in 
    order to provide the top level API.


\end{document}